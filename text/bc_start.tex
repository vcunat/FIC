%%%%%%%%%%%%%%%%%%%%%%%%%%%%%%%%%%%%%%%%%%
%%%                                    %%%
%%% Šablona bakalářské práce na MFF UK %%%
%%%                                    %%%
%%% (c) František Štrupl, 2005         %%%
%%%                                    %%%
%%%%%%%%%%%%%%%%%%%%%%%%%%%%%%%%%%%%%%%%%%

%%% POZOR: Úprava bakalářské práce je závislá rovněž na volbě jednostranného resp. oboustranného tisku.
%%%        Bližši informace naleznete v dokumentu Úprava bakalářské práce, který se nalézá na adrese:
%%%        http://www.mff.cuni.cz/studium/obecne/bplayout/pok12mo4.pdf

%\pagestyle{headings}
%\pagestyle{plain}

%\frenchspacing % aktivuje použití některých českých typografických pravidel


%\csprimeson % zapne jednoduché psaní českých uvozovek pomocí klasických znaků, ale potom pozor 
             % na originální apostrofy, které budou chybně interpretovány!!!

%%% Následuje první, úvodní, strana bakalářské práce. Jednotlivé položky nahraďte dle vlastních
%%% údajů. Změnit podle konkrétní délky jednotlivých položek můžete i zalomení řádků.

\selectlanguage{czech}

\begin{titlepage}
\begin{center}

{ \large
Univerzita Karlova v Praze\\
Matematicko-fyzikální fakulta\\
}

\vspace{10mm}

{\Large\bf BAKALÁŘSKÁ PRÁCE}

\vfill

%%% Aby vložení loga správně fungovalo, je třeba mít soubor logo.eps nahraný v pracovním adresáři,
%%% tj. v adresáři, kde se nachází překládaný zdrojový soubor. Soubor logo.eps je možné získat např.
%%% na adrese: http://www.mff.cuni.cz/fakulta/symboly/logo.eps
\includegraphics[scale=0.35]{logo} 

\vfill

%\normalsize
{\Large \bcAuthor}\\ % doplňte vaše jméno
\vspace{5mm}
{\Large\bf \bcTitle}\\ % doplňte název práce
\vspace{5mm}
\bcKSVI\\ % doplňte název katedry či ústavu
\end{center}
\vfill

\large
\noindent Vedoucí bakalářské práce: \bcTeacher % doplňte odpovídající údaje
%%% další řádek můžete ve většině případů (tj. pokud údaje uvedené výše nejsou příliš dlouhé) zrušit
%\hskip20mm   je-li odlišné od uvedeného názvu katedry či ústavu 
\vspace{1mm} 

\noindent Studijní program: informatika % doplňte odpovídající údaje
%%% další řádek můžete ve většině případů (tj. pokud údaje uvedené výše nejsou příliš dlouhé) zrušit
%\hskip20mm oboru (směru),  příp. název studijního plánu

\vspace{20mm}

\begin{center}
2009 % doplňte rok vzniku vaší bakalářské práce
\end{center}

\end{titlepage} % zde končí úvodní strana

\normalsize % nastavení normální velikosti fontu
\setcounter{page}{2} % nastavení číslování stránek
\cleardoublepage

\noindent Děkuji svému vedoucímu RNDr.~Tomáši Dvořákovi za jeho podporu a rady v~průběhu vzniku bakalářské práce. Dále bych chtěl poděkovat RNDr.~Janě Kalové za to, že mě přivedla
k~tomuto krásnému tématu.

\vspace{\fill} % nastavuje dynamické umístění následujícího textu do spodní části stránky
\noindent Prohlašuji, že jsem svou bakalářskou práci napsal samostatně a výhradně s použitím citovaných pramenů. Souhlasím se zapůjčováním práce a jejím zveřejňováním.

\bigskip
\noindent V Praze dne \today\hspace{\fill}\bcAuthor\\ % doplňte patřičné datum, jméno a příjmení
%%%   Výtisk pak na tomto míste nezapomeňte PODEPSAT!
%%%                                         *********

\cleardoublepage
\tableofcontents % vkládá automaticky generovaný obsah dokumentu

\cleardoublepage % přechod na novou pravou stránku
%%% Následuje strana s abstrakty. Doplňte vlastní údaje.
{ \small
\noindent
{\bf Název práce:} \bcTitle\\
{\bf Autor:} \bcAuthor\\
{\bf Katedra (ústav):} \bcKSVI\\
{\bf Vedoucí bakalářské práce:} \bcTeacher\\
{\bf E-mail vedoucího:} \bcTeacherMail

\paragraph{Abstrakt:} V předložené práci studujeme metody fraktální komprese obrazu.
Jsou zde rozebrány základní používané techniky, publikovaná rozšíření a jsou navržena a implementována drobná vylepšení stávajících metod. Dále je představen modulární systém umožňující vyměňování různých variant jednotlivých částí kompresního procesu a zjednodušující tak porovnání účinnosti různých algoritmů.

Nově navržená vylepšení jsou vyhodnocena na testovacích obrázcích. Představená penalizační metoda dosahuje vyšších kvalit dekomprimovaných obrázků. Diferenční kódování v~kombinaci s~přeuspořádáním cílových bloků snižuje nároky na prostor okolo 5\%. Predikce implementovaná pomocí KD-stromů se naopak ukázala jako nepříliš účinná v~porovnání s~článkem používajícím jinou datovou strukturu.

\paragraph{Klíčová slova:} fraktál, komprese obrazu, systém iterovaných funkcí

%\vspace{10mm}
\vfill % výměna - IMHO vypadá mnohem lépe

\selectlanguage{english}
\noindent
{\bf Title:} \bcTitleEn\\
{\bf Author:} \bcAuthor\\
{\bf Department:} \bcKSVIen\\
{\bf Supervisor:} \bcTeacher\\
{\bf Supervisor's e-mail address:} \bcTeacherMail

\paragraph{Abstract:} In the present work we study fractal image compression.
We discuss basic techniques, published improvements, and a few proposed enhancements of the current methods including their implementation. A~framework for fractal compression is introduced. It is designed in order to be able to replace individual parts of the encoding process by different algorithms, simplifying comparison of their combinations.

Presented enhancements are evaluated on a set of testing images. With proposed penalisation method decompressed images reach higher qualities. The differential coding combined with reordering of range-blocks decreases the amount of needed space approximately by 5\%. On the other hand the prediction implemented by KD-trees isn't as accurate in comparison to a paper using another data structure.

\paragraph{Keywords:} fractal, image compression, iterated function system

% přidáno s vfillem
\vspace{10mm}
}
\selectlanguage{czech}
\cleardoublepage % přechod na novou pravou stránku
\endinput
